% Quarto技術資料用LaTeXプリアンブル
%
% 備考:
%   Markdownファイル用YAMLヘッダー: _metadata_technote.yml
% 更新履歴:
%   20250813:
%     - titlesecパッケージの設定にParagraphのフォーマットを追加
%     - インラインコードブロックの書式を変更
%     - コードブロックの余白を調整し、改行に対応
%     - 不要なプリアンブルを整理
%     - 数式フォントを変更・調整
%   20250810:
%     - バージョン記録開始


% === 必要なパッケージの読み込み ===

\usepackage{array} % tabular環境を拡張、セルの幅を設定できる
\usepackage{multirow} % 表のセルを縦方向に結合

% SI準拠の単位表示
\usepackage{siunitx}
\sisetup{mode = match}  % モード(数式/本文)に合わせて、数字や単位の書式を切替

\usepackage[dvipsnames]{xcolor}  % 色の設定
\usepackage{here}  % {fig-pos="H"}で図をソースコードの位置に強制的に配置
\usepackage{pdfpages}  % 外部で作成したPDF(図面、別章、付録など)をそのまま挿入
\usepackage[hidelinks]{hyperref}  % PDFにハイパーリンク機能を追加

% --- 未使用パッケージ ---
% \renewcommand{\jsParagraphMark}{} %ltjsarticleを使う時用、段落見出しのマークを消す
% \usepackage{tikz}  % LaTeXで図や図形を描くための描画パッケージ
% \usepackage{luatexja-otf}  % LuaLaTeXで日本語を組版する際にOpenTypeフォントを直接利用できるようにする
% \usepackage{unicode-math}
% \usepackage{longtable}


% === lualatex用のフォント設定 ===

\usepackage[no-math]{fontspec}  % LuaLaTeXでOpenType/TrueTypeフォントを直接指定する、数式フォントの設定は無効化
\usepackage{amsmath,amssymb}  % 数式で\align環境などを使えるようにする、追加の数学記号を使えるようにする
\usepackage[noto-jp,deluxe,expert,bold]{luatexja-preset}  % 日本語フォントにプリセットを使う
\ltjsetparameter{ jacharrange = {-2} }  % リテラルなギリシャ文字を欧文扱いする(ギリシャ文字に欧文フォントをそのまま使う)
\setmainfont[Ligatures=TeX]{Noto Sans}  % 英数字メインフォントを設定
\setsansfont[Ligatures=TeX]{Noto Sans}  % 英数字サンセリフフォントを設定
\setmainjfont{Noto Sans JP}  % 日本語メインフォントを設定
\setsansjfont{Noto Sans JP}  % 日本語サンセリフフォントを設定
\setmathfont[Scale=1.0]{LeteSansMath}  % 数式用フォントを設定
\setmonofont{PlemolJP}  % 英数字等角フォントを設定
\setmonojfont{PlemolJP}  % 日本語等角フォントを設定

% --- 未使用フォント ---
% \setmathfont[Scale=1.0]{NewCMSansMath-Regular.otf}  % 数式用フォントを設定
% \setmathfont[Scale=1.0]{texgyrepagella-math.otf}  % 数式用フォントを設定


% === レイアウト調整 ===

% --- ltjsbook用 ---
% \usepackage{layout}
% \setlength{\fullwidth}{\paperwidth-36mm}
% \setlength{\textheight}{0.83\paperheight}
% \addtolength\topmargin{-0.2in}
% \setlength{\headsep}{30pt} % ヘッダーと本文の間隔を20ptに設定

% --- bxjsbook用 ---
\setlength{\fullwidth}{\paperwidth-36mm}
\setlength{\textheight}{0.78\paperheight} % 本文部分の高さ
\addtolength\topmargin{-0.2in}
\setlength{\headsep}{30pt} % ヘッダーと本文の間隔を20ptに設定


% === タイトルのフォーマット ===

% プリアンブルで \maketitle を再定義
\makeatletter
\newcommand{\subtitle}[1]{\def\@subtitle{#1}}
\newcommand{\@subtitle}{}  % 初期値は空
\renewcommand{\maketitle}{
  \begin{center}
    {\bfseries\huge \@title \par}
    \ifx\@subtitle\@empty  % サブタイトルが空なら何もしない
    \else
      {\normalsize \@subtitle \par}
    \vskip 1em  % サブタイトルと著者の間の空白
    {\large \@author \par}
    \vskip 1em  % 著者と日付の間の空白
    {\large \@date \par}
  \end{center}
  \vskip 2em  % 日付と本文の間の空白
}
\makeatother


% === コードブロックのフォーマット ===

% --- インラインコードブロック ---
\definecolor{codebg}{RGB}{234,234,234}  % 明るいグレーの背景色
\let\oldtexttt\texttt
\renewcommand{\texttt}[1]{  % \textttの再定義
  \colorbox{codebg}{  % 背景色
    \textcolor{Maroon}{  % 文字色
      \small  % 文字サイズ
      \oldtexttt{#1}
    }
  }
}

% --- コードブロック ---
\usepackage{fvextra}    % fancyvrbの拡張
\setlength{\fboxrule}{6pt}  % 枠の太さを太くする
\fvset{
  bgcolor = codebg,
  frame = single,
  framesep = 0pt,        % 枠と内容の間の余白
  rulecolor = \color{codebg},
  fontsize = \small,
  breaklines = true,
  breakanywhere = true,  % 単語の途中でも改行する
  commandchars = \\\{\},  % Pandocの構文ハイライトマクロに必要
}
\usepackage{framed} % <- Shaded は framed の shaded/snugshade を使う
\definecolor{shadecolor}{RGB}{255,255,255} % 外枠の Shaded を白に


% === Headerのフォーマット ===
\usepackage{fancyhdr}
\usepackage{lastpage}
\pagestyle{fancy}
\fancyhf{}
\fancyhead[RO,RE]{\normalfont \thepage /\pageref*{LastPage}} % ヘッダー右にページ番号
% ヘッダー左に文書タイトル
\let\origtitle\title 
\renewcommand{\title}[1]{\lhead{\gt\sf\bfseries {#1}}\origtitle{#1}}


% === Headingsのフォーマット ===
\usepackage{titlesec}  % 見出しのフォーマット用パッケージ

% --- Section ---
\titleformat
{\section} % command
[block] % shape
{\color{NavyBlue}\gt\sf\bfseries\Large} % format
{
  \definecolor{teal}{gray}{0.30}
  \begin{picture}(-11,0)
    % \put(-10,-5){
    %   \begin{tikzpicture}
    %     \fill[teal] (0pt,0pt) rectangle (5pt,19pt);
    %   \end{tikzpicture}
    % }
    \put(-11,-5){
      \color{NavyBlue}
      \line(1,0){\hsize}
    }
  \end{picture}
  \Large%
  \hspace{-4pt}
  \thesection .
  \hspace{0pt}
} % label
{0pt} % sep
{} % before-code
% [] % after-code

% --- Section (.unnumbered) ---
\titleformat
{name=\section,numberless} % command
[block] % shape
{\color{NavyBlue}\gt\sf\bfseries\Large} % format
{} % label
{0pt} % sep
{
  \definecolor{teal}{gray}{0.30}
  \begin{picture}(-11,0)
    % \put(-10,-5){
    %   \begin{tikzpicture}
    %     \fill[teal] (0pt,0pt) rectangle (5pt,19pt);
    %   \end{tikzpicture}
    % }
    \put(-11,-5){
      \color{NavyBlue}
      \line(1,0){\hsize}
    }
  \end{picture}
} % before-code
% [] % after-code

% --- Subsection ---
\titleformat
{\subsection} % command
[block] % shape
{\color{NavyBlue}\gt\sf\bfseries\large} % format
{\thesubsection .} % label
{8pt} % sep
{} % before-code

% --- Subsubsection ---
\titleformat
{\subsubsection} % command
[block] % shape
{\gt\sf\bfseries} % format
{} % label
{} % sep
{} % before-code

% --- Paragraph ---
\titleformat
{\paragraph} % command
[runin] % shape
{\normalsize\gt\sf\bfseries} % format
{} % label
{} % sep
{} % befor-code
% スペーシングの調整
\titlespacing*{\paragraph}{0pt}{3.25ex plus 1ex minus .2ex}{1em}
